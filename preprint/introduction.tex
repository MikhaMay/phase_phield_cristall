%
% introduction.tex
%

\section{Введение}

В данной работе рассматривается моделирование кристаллических структур с использованием методов прямого моделирования и фазового поля. Одномерные кристаллические решётки представляют собой системы, позволяющие исследовать фундаментальные аспекты кристаллообразования и динамики фазовых переходов.

В моделях фазового поля кристаллическая структура описывается заданной в пространстве гладкой функцией $\phi = \phi(x, t)$. Она может быть интерпретирована как <<вероятность>> нахождения частицы в момент времени $t$ в точке пространства $x$.

Таким образом, для кристаллического тела $\phi$ имеет вид периодической функции, максимумы которой соответствуют равновесным положениям атомов в кристаллической решетке. Для жидкостей фазовое поле $\phi$ имеет распределение, близкое к постоянному.

Для стационарного состояния $\phi$ доставляет минимум специально выбранному функционалу свободной энергии $\Psi$, который может быть построен на основе потенциала взаимодействия атома. Подходящий выбор функционала обеспечивает формирование, в равновесии, кристаллической решетки с заданными свойствами.

В неравновесном случае эволюция $\phi$ определяется эволюционным уравнением, которое выводится из вида функционала $\Psi$. В процессе эволюции фазовое поле $\phi$ сохраняет свою общую фазовую концентрацию~--- интеграл от $\phi$ по всей области пространства остаётся неизменным.


Для моделирования кристаллов фазового поля в одномерном случае с функцией $\Psi = \Psi(\phi, \nabla \phi, \Delta \phi)$, где $\phi (x, t)$~--- фазовое поле, а $\nabla \phi$ и $\Delta \phi$ обозначают его градиент и лапласиан соответственно, можно использовать систему уравнений, основанную на уравнении Гинзбурга-Ландау или на более общем функционале свободной энергии. Равновесное состояние системы соответствует минимуму этого функционала.

Ключевой задачей моделирования является анализ устойчивости и динамики изменения фазовых состояний, что включает в себя решение уравнения эволюции фазового поля и исследование зависимости свойств системы от параметров модели.

В данной работе проводится численное исследование простейших моделей фазового поля, при этом особое внимание уделяется анализу динамических характеристик системы в различных начальных и граничных условиях. Это позволяет оценить, как эти условия влияют на процесс формирования кристаллических структур. Работа включает в себя подробное описание используемого математического аппарата и методов численного моделирования, а также представление и анализ результатов моделирования.

\endinput
% EOF